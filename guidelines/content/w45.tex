\chapter*{Week 45}








\section*{Week Goals}

Starting from this week, we can follow the project plan that you have designed.
In these instruction sheets, I will provide you some guidlines, resources and technical details on how to achieve the weekly goals.
This week, the project plan features four main goals:
\begin{itemize}
\item go through the \ROS{} tutorials;
\item assemble the remaining hardware;
\item connect to the \CF{}s;
\item set up a \texttt{github} repository.
\end{itemize}
%I would like to add one:
%\begin{itemize}
%  \item update the firmware on the \CF{}s and on the \texttt{Loco Positioning node}s.
%\end{itemize}
%Updating the firmware is important to get the latest features and bug fixes from \texttt{Bitcraze}.
%I will give you detailed instructions on how to do these updates.
I took the liberty to add something extra that is closely related, so that you don't have to do it later.
Let's now go through each of the goals.








\section*{Go through the \ROS{} tutorials}

I would recommend that you go through the \href{http://wiki.ros.org/ROS/Tutorials#Beginner_Level}{beginner tutorials}; we can leave out the intermediate tutorials for now.
You can start off from where you stopped at the \ROS{} tutorial session that we had on Nov 4th.
From now on, I will assume that you have installed \ROS{} and created a catkin workspace named \verb|catkin_ws| in your home folder.






\section*{Assemble the remaining hardware}

For the moment being, we can substitute the battery holder with a \texttt{Loco Positioning deck} on each of the \CF{}s.
The \texttt{Loco Positioning deck} is considered an expansion deck; therefore, its mounting is covered in the \href{https://www.bitcraze.io/getting-started-with-expansion-decks/}{tutorial on expansion decks}.
Changing the battery holder with the \texttt{Loco Positioning deck} only takes a few seconds.
Just make sure that you mount it with the right orientation, as explained in the tutorial.

The deck should turn on when you turn on the \CF{}, because it gets power directly from the pins.
If you turn on the \CF{}, but the deck does not turn on, it is probably because we need to update the \CF{}'s firmware.
Here you can find \href{https://github.com/bitcraze/crazyflie-firmware}{a guide to update the \CF{} firmware}.
Note: updating the firmware of the \CF{} may turn out to be a little hard, because you may need to install some dependency.
If you get an error message, you best shot is to google it.
Chances are that the issue has already been experienced, and there is a quick fix.
If you get lost, call me up and I will also try to sort it out.

After updating the firmware, the \texttt{Loco Positioning deck} should turn on and off together with the \CF{}.

Soon enough, we will need to also update the firmware on the six \texttt{Loco Positioning node}s.
I think that it would be best to get started with that already now.
Here you can find \href{https://github.com/bitcraze/lps-node-firmware}{a guide to update the \texttt{Loco Positioning node} firmware}.

The \texttt{Loco Positioning node}s have three different functioning modes: \emph{anchor}, \emph{tag} and \emph{sniffer}.
After updating the firmware, we can set up our \texttt{Loco Positioning node} to behave as anchors.
Here we have the \href{https://wiki.bitcraze.io/projects:lps:node#configuring_the_node}{instructions for setting the \texttt{Loco Positioning node}s as anchors}.

Speaking of assembling hardware, later on we will need to find a way to position the six \texttt{Loco Positioning nodes} in the project room in such a way that they do not move.
To do this, we can use rubber bands (quick and dirty) or get some wall mounts (takes more time, but more stable and elegant).
It should be relatively easy to get us some wall mounts.
Let's discuss this on our meeting, Monday Nov 7th.









\section*{Safety}

In the next part, we are going to make the propellers spin.
Therefore, it is probably a good time to talk some safety.
We already discussed something on our first group meeting, but it's good to remind ourselves.
Please, watch out whenever a \CF{} is turned on.
Compared to other drones (even those for recreatinal use) the \CF{} is relatively safe, due to the small size.
Still, if you touch it while the propellers are spinning, you may get a small cut.
More importantly, you never want a \CF{}s to get anywhere near you eyes.
Keep your eyes away from the \CF{}s.
% And, just to be sure, keep your eyes away from the \CF{}s.
% Did I tell you to keep your eyes away from the \CF{}s?
% Good, because it's important to keep your eyes away from the \CF{}s.







\section*{Connect to the \CF{}s}

The simplest way to communicate with the \CF{}s would be to use the \texttt{Bitcraze} virtual machine.
However, since we want to design a path planner \ROS{} node for the \CF{}s (and not just fly them with a joystick) we actually need to learn to connect to them through \ROS{}.

If you don't have them yet, download the \ROS{} libraries for the \CF{}s:
go to \href{}{Wolfgang Hoenig's repository}, and clone it in your \verb|catkin_ws/src| folder.
You can do it from the command line, by typing the following lines.
Remember: the dollar sign \verb|$| marks the beginning of a command, you do not type it yourself.

\begin{Verbatim}[fontsize=\small]

$ cd ~/catkin_ws/src
$ git clone https://github.com/whoenig/crazyflie_ros.git

\end{Verbatim}

Now you need to compile and source your catkin workspace.
``Compile and source'' means that you compile the code that you have just downloaded and inform \ROS{} that the compiled files are there, ready to be used.
You can do it as follows.

\begin{Verbatim}[fontsize=\small]

$ cd ~/catkin_ws
$ catkin_make
$ source ./devel/setup.bash

\end{Verbatim}

Now you need to set permission for non-root users to access the USB ports, otherwise \ROS{} will not have access to the \texttt{Crazyradio PA}.
Here are the \href{https://github.com/bitcraze/crazyflie-lib-python#linux}{instructions for setting USB permissions}.

Now plug the \texttt{Crazyradio PA} in an USB port, and turn on a \CF{}.
Then type the following.

\begin{Verbatim}[fontsize=\small]

$ rosrun crazyflie_tools scan

\end{Verbatim}

If everything goes well, this will print the radio address of the \CF{}.
You can expect something like that

\begin{Verbatim}[fontsize=\small]

Configured Dongle with version 0.53
radio://0/80/250K

\end{Verbatim}

To connect to the \CF{}, you can type the following.

\begin{Verbatim}[fontsize=\small]

$ roslaunch crazyflie_driver crazyflie_server.launch

\end{Verbatim}

Then, open another terminal and type

\begin{Verbatim}[fontsize=\small]

$ roslaunch crazyflie_driver crazyflie_add.launch uri:=radio://x/xx/xxx

\end{Verbatim}

where \verb|x/xx/xxx| must be replaced with the address that you received from \verb|rosrun crazyflie_tools scan|.


You should now be connected to the \CF{}.
You can check if you are connected by printing the active \ROS{} nodes.

\begin{Verbatim}[fontsize=\small]
$ rosnode list
\end{Verbatim}

You should see something like this.

\begin{Verbatim}[fontsize=\small]
/crazyflie_server
/rosout
\end{Verbatim}

You can also check the \ROS{} topics that these nodes are using.

\begin{Verbatim}[fontsize=\small]
$ rostopic list
\end{Verbatim}

You should see something like this.

\begin{Verbatim}[fontsize=\small]
/battery
/cmd_vel
/imu
/magnetic_field
/pressure
/rosout
/rosout_agg
/rssi
/temperature
\end{Verbatim}

You can also print the content of these topics to screen.
For example, you can type the following.

\begin{Verbatim}[fontsize=\small]
$ rostopic echo /battery
\end{Verbatim}

Now we can make the propellers spin.
To do this, you have to publish a message on the topic \verb|/cmd_vel|, which is a command to the pilot of the \CF{}.
You only need to set the \texttt{linear.z} attribute of the message, and you can leave the other to zero.
I suggest you set it to something between \texttt{10000} and \texttt{15000}, which will start the propellers without making the \CF{} take off.
However, this may vary with the \CF{} or with the battery level, so it's better if you hold the \CF{} in your hands.
Hold the \CF{} from below, so that your fingers are away from the propellers, and keep it far from your eyes.

You can publish from the command line with the command \texttt{rospub}.
Here is an example.

\begin{Verbatim}[fontsize=\small]
$ rostopic pub /cmd_vel geometry_msgs/Twist "linear:
        x: 0.0
        y: 0.0
        z: 13000
angular:
        x: 0.0
        y: 0.0
        z: 0.0" -r 10
\end{Verbatim}

You can stop the publishing (and hence the propeller spinning) by pressing \verb|Ctrl-C| in the terminal.

An alternative is to use \href{http://wiki.ros.org/rqt_publisher}{the \texttt{Message Publisher} tool from \texttt{rqt}}.






\section*{Connect to the \LPN{}s}

At this point, you should also be able to set up a connection with the \LPN{}s, and printing range measurements (that is, the distance between the anchor and the \CF{}) on you sreen.
Let's try that out.

Download the \ROS{} library for the \LPN{}s.
You can do it by command line as follows.

\begin{Verbatim}[fontsize=\small]
$ cd ~/catkin_ws/src
$ git clone https://github.com/bitcraze/lps-ros
\end{Verbatim}

Now you need to compile and source your workspace again

\begin{Verbatim}[fontsize=\small]
$ cd ~/catkin_ws
$ catkin_make
$ source ./devel/setup.bash
\end{Verbatim}

Make sure that the \texttt{Crazyradio PA} is still plugged in, and turn on the \CF{}.
Connect at least one \LPN{} to a power supply; connecting it to the power supply will automatically turn it on.
Then, launch the position estimator as follows

\begin{Verbatim}[fontsize=\small]
$ roslaunch bitcraze_lps_estimator dwm_loc.launch uri:=radio/x/xx/xxx
\end{Verbatim}

where again \verb|x/xx/xxx| is the address of your \CF{}.
Now type the following.

\begin{Verbatim}[fontsize=\small]
$ rostopic echo /crazyflie/log_ranges
\end{Verbatim}

Do you get the range measurements?




\section*{Create a project package}

You probably already have created a \texttt{catkin} package for the project; if not, you can follow these steps.
I think that the best way would be that you only do this on one computer; then everybody clones the package with \texttt{git}.

\begin{Verbatim}[fontsize=\small]
$ cd ~/catkin_ws/src
$ catkin_create_pkg el2425-bitcraze-group std_msgs geometry_msgs rospy
\end{Verbatim}

Of course \texttt{el2425-bitcraze-group} is only an example name;
you can choose the name that you like best.
The other names in the command are the dependencies of the package.
For now we can put only \verb|std_msgs|, \verb|geometry_msgs|, and \verb|rospy|;
we can add more later if needed.

I suggest that you have the following structure in your folder:

\begin{itemize}
  \item \verb|msg| - a subfolder with \ROS{} messages; they are text files with \verb|.msg| extension;
  \item \verb|srv| - a subfolder with \ROS{} services; they are text files with \verb|.srv| extension;
  \item \verb|launch| - a subfolder with \ROS{} launchfiles; they are text files with \verb|.xml| extension;
  \item \verb|scripts| - a subfolder with the code implementing \ROS{} nodes; for example, they can be text files \verb|.py| or \verb|.cpp| extension;
  \item \verb|src| - a subfolder with all the other code.
\end{itemize}







\section*{Setting up a \texttt{git} repository}

If you need to familiarize with \texttt{git}, here you have a nice \href{https://try.github.io/levels/1/challenges/1}{\texttt{git} tutorial}.

The git repository should coincide with the \texttt{catkin} package that you have created to host the project code.

Once you have set up the repository on your computer, you have to upload it to some hosting server, such as \texttt{github} or \texttt{gitlab}.
You can pick your favorite hosting server.

Once you have uploaded the repository, please send me a link to join it as a contributor, so that I can follow along.




\section*{Have fun}

This week may be one of the most technical, especially because of the firmware update.
I hope you still have a lot of fun.
See you on \deadline{Monday, Nov 7th at 16:00} in the Automatic Control department.
